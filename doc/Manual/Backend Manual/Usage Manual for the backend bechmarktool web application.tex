\documentclass[11pt,a4paper]{article}
\usepackage{url}
\usepackage{color}
\usepackage{listings}
\usepackage{color}
 
\definecolor{codegreen}{rgb}{0,0.6,0}
\definecolor{codegray}{rgb}{0.5,0.5,0.5}
\definecolor{codepurple}{rgb}{0.58,0,0.82}
\definecolor{backcolour}{rgb}{0.95,0.95,0.92}
 
\lstdefinestyle{mystyle}{
    backgroundcolor=\color{backcolour},   
    commentstyle=\color{codegreen},
    keywordstyle=\color{magenta},
    numberstyle=\tiny\color{codegray},
    stringstyle=\color{codepurple},
    basicstyle=\footnotesize,
    breakatwhitespace=false,         
    breaklines=true,                 
    captionpos=b,                    
    keepspaces=true,                 
    numbers=left,                    
    numbersep=5pt,                  
    showspaces=false,                
    showstringspaces=false,
    showtabs=false,                  
    tabsize=2
}
 
\lstset{style=mystyle}
\lstset{language=Python} 
\title{Usage Manual for the backend bechmarktool web application}
\renewcommand*\contentsname{Summary}
\let\OldUrlFont\UrlFont \renewcommand{\UrlFont}{\small\OldUrlFont}
\begin{document}
\maketitle
\tableofcontents

\section{Introduction}
This manual is a quick guide that explains how to manipulate data with the benchmarktool web application.
It explains how to check the existing registers on the database, and how to add new benchmark sets and benchmark parameters. 


\section{Listing the existent registers on a table}
To list all the registers on an existing table of the system, you must perform perform a GET on the following URL:

\noindent\url{https://<host_name>/<application name>/default/api/<table_name>.json}

In the current deployment scenario:

\noindent\url{https://gwt.eit.uni-kl.de/benchmarktool/default/api/<table_name>.json}

\section{Adding new register to tables}
\subsection{Adding an option price}
A register in the option price table is composed by an ID and a name. 
Since the ID fields are auto-incremented by default in our application, only the name should be defined. 
The type of the field "name" is string.

To add a new option price it is necessary to insert a tuple in the database.
In our architecture, the way of doing this is performing a POST via HTTPS, since we are using a REST architecture.
The data to be added should be in the payload field and then a POST request has to be sent to the URL: 

\noindent\url{https://gwt.eit.uni-kl.de/benchmarktool/default/api/option_price.json}

In case of a new installation, the URL format should be:

\noindent\url{https://<host_name>/<application name>/default/api/option_price.json}

An example using the requests library for Python:
\begin{lstlisting}
payload_option_price={"name":"OPTION_TYPE_CALL_DIGITAL"}
requests.post("https://gwt.eit.uni-kl.de/benchmarktool/default/api/option_price.json", data=payload_option_price)
\end{lstlisting}

\subsection{Adding a barrier type}
A register in the barrier table is composed by an ID and a name. 
Since the ID fields are auto-incremented by default in our application, only the name should be defined. 
The type of the field "name" is string.
To add a new barrier type it is necessary to insert a tuple in the database. 
In our architecture, the way of doing this is performing a POST via HTTPS, since we are using a REST architecture.
The data to be added should be in the payload field and then a POST request has to be sent to the URL: 

\noindent\url{https://gwt.eit.uni-kl.de/benchmarktool/default/api/barrier.json}

In case of a new installation, the URL format should be:

\noindent\url{https://<host_name>/<application name>/default/api/barrier.json}

An example using the requests library for Python:
\begin{lstlisting}
	payload_barrier={"name":"BARRIER_TYPE_KNOCK_OUT"}
	requests.post("https://gwt.eit.uni-kl.de/benchmarktool/default/api/barrier.json", data=payload_barrier)
\end{lstlisting}

\subsection{Adding a market parameter}
A market parameter is a tuple composed by the parameters that define a specific market.
It is composed by the following parameters:
\begin{itemize}
	\item correlation
	\item long run variance
	\item speed of revertion
	\item volatility of volatility
	\item spot price
	\item spot volatility
	\item riskless interest rate
\end{itemize}

To perform a simulation you must define all those parameters, since they are a fundamental part of the market simulation algorithms.
The way to define them is to add a tuple in the database by performing a POST via HTTPS, since we are using a REST architecture.
The data to be added should be in the payload field and then a POST request has to be sent to the URL: 

\noindent\url{https://gwt.eit.uni-kl.de/benchmarktool/default/api/market_parameters.json}

In case of a new installation, the URL format should be:

\noindent\url{https://<host_name>/<application name>/default/api/market_parameters.json}

An example using the requests library for Python:
\begin{lstlisting}
payload_market_param={"spot_price": 100.0, "long_run_variance": 0.09, "volatility_of_volatility": 1.0, "spot_volatility": 0.09, "riskless_interest_rate": 0.05, "correlation": -0.3, "speed_of_revertion": 2.0}
requests.post("https://gwt.eit.uni-kl.de/benchmarktool/default/api/market_parameters.json", data=payload_market_param)
\end{lstlisting}

The respective types of the parameters are:
\begin{itemize}
	\item correlation: double
	\item long\_run\_variance: double
	\item speed\_of\_revertion: double
	\item volatility\_of\_volatility: double
	\item spot\_price: double
	\item spot\_volatility: double
	\item riskless\_interest\_rate: double
\end{itemize}

\subsection{Adding an option parameter}

An option parameter is a tuple composed by the parameters that define an specific option.
It is composed by the following parameters:
\begin{itemize}
	\item option type
	\item strike price
	\item time to maturity
	\item lower barrier type
	\item lower barrier value
	\item upper barrier type
	\item upper barrier value
\end{itemize}

To perform a simulation you must define all those parameters, since they are a fundamental part of the benchmarks.
The way to define them is to add a tuple in the database by performing a POST via HTTPS, since we are using a REST architecture.
The data to be added should be in the payload field and then a POST request has to be sent to the URL: 

\noindent\url{https://gwt.eit.uni-kl.de/benchmarktool/default/api/option_parameters.json}

In case of a new installation, the URL format should be:

\noindent\url{https://<host_name>/<application name>/default/api/option_parameters.json}

An example using the requests library for Python:
\begin{lstlisting}
payload_option_param = {"time_to_maturity": 1.0, "strike_price": 100.0,  "lbarrier_value": 90.0, "ubarrier_value": 110.0, "option_type": 2, "lbarrier_type": 1, "ubarrier_type": 1}
requests.post("https://gwt.eit.uni-kl.de/benchmarktool/default/api/option_parameters.json", data=payload_option_param)
\end{lstlisting}


The respective types of the parameters are:
\begin{itemize}
	\item option\_type: integer (FK to table option\_price)
	\item strike\_price: double
	\item time\_to\_maturity: double
	\item lbarrier\_type: integer (FK to table barrier)
	\item lbarrier\_value: double
	\item ubarrier\_type: integer (FK to table barrier)
	\item ubarrier\_value: double
\end{itemize}

\subsection{Benchmark Set}
A benchmark set is composed by name, description, market parameters and option parameters. 
As it is an REST architecture, to add a new benchmark set it is necessary to perform a POST  via HTTPS.
The data to be added should be in the payload field and then a POST request has to be sent to the URL: 

\noindent\url{https://gwt.eit.uni-kl.de/benchmarktool/default/api/benchmark_set.json}

In case of a new installation, the URL format should be:

\noindent\url{https://<host_name>/<application name>/default/api/benchmark_set.json}

An example using the requests library for Python:
\begin{lstlisting}
payload_benchmark_set = {"mkt_parameters": 1, "description": "Benchmark set detailed description.", "opt_parameters": 1, "name": "Benchmark 1"}
requests.post("https://gwt.eit.uni-kl.de/benchmarktool/default/api/benchmark_set.json", data=payload_benchmark_set)
\end{lstlisting}

The respective types of the parameters are:
\begin{itemize}
	\item name: string
	\item description: long string
	\item mkt\_parameters: (FK to table option\_parameters)
	\item opt\_parameters: (FK to table market\_parameters)
\end{itemize}

\section{Starting a Working Node}
Working nodes are automatically registered into the system once the web2py process is started. 
To be able to disconnect from the remote connection, it is advisable to first start a screen and then start web2py inside it.
The words inside the curly brackets must be replaced with the appropriated ones.
\begin{lstlisting}
[web2py: ~]# screen
[web2py: ~]# python {full path to web2py}/web2py.py -a '<recycle>' -i {interface's IP} -p {port}
web2py Web Framework
Created by Massimo Di Pierro, Copyright 2007-2015
Version 2.9.12-stable+timestamp.2015.01.17.06.11.03
Database drivers available: sqlite3, pymysql, pg8000, imaplib

please visit:
        http://127.0.0.1:8000/
use "kill -SIGTERM 19414" to shutdown the web2py server

starting scheduler for "benchmarktool"...

Currently running 1 scheduler processes
Processes started

\end{lstlisting}


\end{document}