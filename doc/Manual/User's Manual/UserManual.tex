\documentclass[11pt,a4paper]{article}
\usepackage{graphicx}           % pacote para importar figuras
\graphicspath{ {images/} }
\usepackage{url}
\usepackage{color}
\usepackage{listings}
\usepackage{color}
 
\definecolor{codegreen}{rgb}{0,0.6,0}
\definecolor{codegray}{rgb}{0.5,0.5,0.5}
\definecolor{codepurple}{rgb}{0.58,0,0.82}
\definecolor{backcolour}{rgb}{0.95,0.95,0.92}
 
\lstdefinestyle{mystyle}{
    backgroundcolor=\color{backcolour},   
    commentstyle=\color{codegreen},
    keywordstyle=\color{magenta},
    numberstyle=\tiny\color{codegray},
    stringstyle=\color{codepurple},
    basicstyle=\footnotesize,
    breakatwhitespace=false,         
    breaklines=true,                 
    captionpos=b,                    
    keepspaces=true,                 
    numbers=left,                    
    numbersep=5pt,                  
    showspaces=false,                
    showstringspaces=false,
    showtabs=false,                  
    tabsize=2
}
 
\lstset{style=mystyle}
\lstset{language=Python} 
\title{Usage Manual for the Bechmarktool Web Application User's Interface}
\renewcommand*\contentsname{Summary}
\let\OldUrlFont\UrlFont \renewcommand{\UrlFont}{\small\OldUrlFont}
\begin{document}
\maketitle
\tableofcontents

\section{Introduction}
This manual is a quick guide that explains how to perform simulation on the benchmarktool web application.
The main features are explained in detail, from the registration steps upto checking the final simulation results.

\section{Accessing the system}
The system can be accessed through the following URL:
\noindent\url{\TODO{add the URL}}
On the left hand side it is possible to check the available options and on the main frame it's shown the content of the active option.

TODO{add initial page image}

\section{Registering on the system}
To start a new simulation, it's necessary to be registered on the system. 
This is done by filling up a simple registration form with personal data.
\begin{figure}[h]
	\begin{center}
		\includegraphics[width=\textwidth]{registration}
		\caption{Registration Form\label{fig:registration}}
	\end{center}
\end{figure}

Figure ~\ref{fig:registration} shows how this form looks like.
After filling up the form, you should click on the ``Save" button and then the registration process is done.

\section{Performing Simulations}
Whenever one wants to compare two different implementations of market simulation with hardware acceleration, one should perform a benchmark over those platforms, through simulations. 
It is necessary to login into the system in order to be able to access the simulation form.
On the left hand side from the menu, the user must select ``Simulator" and a screen as in the image below will be shown:

\begin{figure}[h]
	\begin{center}
		\includegraphics[width=\textwidth]{simulator_login}
		\caption{Login Form\label{fig:login}}
	\end{center}
\end{figure}

To login, it is necessary to fill in the form with correct username - password, as well as the CAPTCHA value, and click the button ``Login".
Then, the simulator interface will be available, allowing the user to perform new simulations.

\begin{figure}[h]
	\begin{center}
		\includegraphics[width=\textwidth]{simulator}
		\caption{Simulation Form\label{fig:simulator}}
	\end{center}
\end{figure}

Figure~\ref{fig:simulator} shows the form where the user can choose the parameters to perform a benchmark.
The dropdown menu ``Benchmark Set" contains all the benchmark sets available in the system.
A benchmark set has a name associated with market parameters and option parameters, meaning that, once a benchmark set is selected, the market and the option parameters are automatically selected as well.
The algorithm parameters are filled already with the default values, but the user has freedom to change them before starting the simulation.
The algorithms parameters are composed by:
\begin{itemize}
	\item Start Level
	\item Multilevel Constant
	\item Epsilon
	\item Number Of Paths On the First Level
	\item Reference Price
	\item Price Precision
\end{itemize}

At the end of the Simulator page, there is a field called ``Available Resources" where the user can select one or more resources to perform the benchmark and compare the results. 
To select a node to perform the simulation it is enough to check the check box beside its name.
Once all the parameters are defined, it is necessary to click on the button ``Start Simulation".
As soon as the simulation finishes, an e-mail will be send to the user with the respective simulation ID.

\section{Checking the Simulation Results}
Whenever a simulation is finished, the user will receive an e-mail with an identification number associated with the simulation.
The simulation results can be viewed from the ``Results" menu.

\begin{figure}[h]
	\begin{center}
		\includegraphics[width=\textwidth]{results}
		\caption{Results Form\label{fig:results}}
	\end{center}
\end{figure}
The results form (Figure~\ref{fig:results}) is a really simple form, where the user can enter the ID of already performed simulations and search for its results. 
After clicking on ``Show Results", a graph comparing both simulations will be displayed.

\begin{figure}[h]
	\begin{center}
		\includegraphics[width=\textwidth]{results_graphs}
		\caption{Results of performed simulations\label{fig:results_graphs}}
	\end{center}
\end{figure}

\end{document}